% Options for packages loaded elsewhere
% Options for packages loaded elsewhere
\PassOptionsToPackage{unicode}{hyperref}
\PassOptionsToPackage{hyphens}{url}
\PassOptionsToPackage{dvipsnames,svgnames,x11names}{xcolor}
%
\documentclass[
  letterpaper,
  DIV=11,
  numbers=noendperiod]{scrartcl}
\usepackage{xcolor}
\usepackage{amsmath,amssymb}
\setcounter{secnumdepth}{-\maxdimen} % remove section numbering
\usepackage{iftex}
\ifPDFTeX
  \usepackage[T1]{fontenc}
  \usepackage[utf8]{inputenc}
  \usepackage{textcomp} % provide euro and other symbols
\else % if luatex or xetex
  \usepackage{unicode-math} % this also loads fontspec
  \defaultfontfeatures{Scale=MatchLowercase}
  \defaultfontfeatures[\rmfamily]{Ligatures=TeX,Scale=1}
\fi
\usepackage{lmodern}
\ifPDFTeX\else
  % xetex/luatex font selection
\fi
% Use upquote if available, for straight quotes in verbatim environments
\IfFileExists{upquote.sty}{\usepackage{upquote}}{}
\IfFileExists{microtype.sty}{% use microtype if available
  \usepackage[]{microtype}
  \UseMicrotypeSet[protrusion]{basicmath} % disable protrusion for tt fonts
}{}
\makeatletter
\@ifundefined{KOMAClassName}{% if non-KOMA class
  \IfFileExists{parskip.sty}{%
    \usepackage{parskip}
  }{% else
    \setlength{\parindent}{0pt}
    \setlength{\parskip}{6pt plus 2pt minus 1pt}}
}{% if KOMA class
  \KOMAoptions{parskip=half}}
\makeatother
% Make \paragraph and \subparagraph free-standing
\makeatletter
\ifx\paragraph\undefined\else
  \let\oldparagraph\paragraph
  \renewcommand{\paragraph}{
    \@ifstar
      \xxxParagraphStar
      \xxxParagraphNoStar
  }
  \newcommand{\xxxParagraphStar}[1]{\oldparagraph*{#1}\mbox{}}
  \newcommand{\xxxParagraphNoStar}[1]{\oldparagraph{#1}\mbox{}}
\fi
\ifx\subparagraph\undefined\else
  \let\oldsubparagraph\subparagraph
  \renewcommand{\subparagraph}{
    \@ifstar
      \xxxSubParagraphStar
      \xxxSubParagraphNoStar
  }
  \newcommand{\xxxSubParagraphStar}[1]{\oldsubparagraph*{#1}\mbox{}}
  \newcommand{\xxxSubParagraphNoStar}[1]{\oldsubparagraph{#1}\mbox{}}
\fi
\makeatother


\usepackage{longtable,booktabs,array}
\usepackage{calc} % for calculating minipage widths
% Correct order of tables after \paragraph or \subparagraph
\usepackage{etoolbox}
\makeatletter
\patchcmd\longtable{\par}{\if@noskipsec\mbox{}\fi\par}{}{}
\makeatother
% Allow footnotes in longtable head/foot
\IfFileExists{footnotehyper.sty}{\usepackage{footnotehyper}}{\usepackage{footnote}}
\makesavenoteenv{longtable}
\usepackage{graphicx}
\makeatletter
\newsavebox\pandoc@box
\newcommand*\pandocbounded[1]{% scales image to fit in text height/width
  \sbox\pandoc@box{#1}%
  \Gscale@div\@tempa{\textheight}{\dimexpr\ht\pandoc@box+\dp\pandoc@box\relax}%
  \Gscale@div\@tempb{\linewidth}{\wd\pandoc@box}%
  \ifdim\@tempb\p@<\@tempa\p@\let\@tempa\@tempb\fi% select the smaller of both
  \ifdim\@tempa\p@<\p@\scalebox{\@tempa}{\usebox\pandoc@box}%
  \else\usebox{\pandoc@box}%
  \fi%
}
% Set default figure placement to htbp
\def\fps@figure{htbp}
\makeatother





\setlength{\emergencystretch}{3em} % prevent overfull lines

\providecommand{\tightlist}{%
  \setlength{\itemsep}{0pt}\setlength{\parskip}{0pt}}



 


\usepackage{amsmath}
\KOMAoption{captions}{tableheading}
\makeatletter
\@ifpackageloaded{caption}{}{\usepackage{caption}}
\AtBeginDocument{%
\ifdefined\contentsname
  \renewcommand*\contentsname{Table of contents}
\else
  \newcommand\contentsname{Table of contents}
\fi
\ifdefined\listfigurename
  \renewcommand*\listfigurename{List of Figures}
\else
  \newcommand\listfigurename{List of Figures}
\fi
\ifdefined\listtablename
  \renewcommand*\listtablename{List of Tables}
\else
  \newcommand\listtablename{List of Tables}
\fi
\ifdefined\figurename
  \renewcommand*\figurename{Figure}
\else
  \newcommand\figurename{Figure}
\fi
\ifdefined\tablename
  \renewcommand*\tablename{Table}
\else
  \newcommand\tablename{Table}
\fi
}
\@ifpackageloaded{float}{}{\usepackage{float}}
\floatstyle{ruled}
\@ifundefined{c@chapter}{\newfloat{codelisting}{h}{lop}}{\newfloat{codelisting}{h}{lop}[chapter]}
\floatname{codelisting}{Listing}
\newcommand*\listoflistings{\listof{codelisting}{List of Listings}}
\makeatother
\makeatletter
\makeatother
\makeatletter
\@ifpackageloaded{caption}{}{\usepackage{caption}}
\@ifpackageloaded{subcaption}{}{\usepackage{subcaption}}
\makeatother
\usepackage{bookmark}
\IfFileExists{xurl.sty}{\usepackage{xurl}}{} % add URL line breaks if available
\urlstyle{same}
\hypersetup{
  pdftitle={Problem Set 5},
  pdfauthor={Brynn Woolley},
  colorlinks=true,
  linkcolor={blue},
  filecolor={Maroon},
  citecolor={Blue},
  urlcolor={Blue},
  pdfcreator={LaTeX via pandoc}}


\title{Problem Set 5}
\author{Brynn Woolley}
\date{}
\begin{document}
\maketitle


\subsection{Problem 1 - OOP
Programming}\label{problem-1---oop-programming}

Create a class to represent Wald-style normal approximation Confidence
Intervals. Do this using S4.

\begin{enumerate}
\def\labelenumi{\arabic{enumi}.}
\item
  For the \texttt{waldCI} class, define the following:

  \begin{enumerate}
  \def\labelenumii{\arabic{enumii}.}
  \tightlist
  \item
    A constructor, which takes in a confidence level (0,1) and either a
    mean and standard error, or a lower and upper bound. This should be
    a custom constructor, not \texttt{new()} or \texttt{waldCI()}.\\
  \item
    A validator.\\
  \item
    A \texttt{show} method.\\
  \item
    Accessors: \texttt{lb}, \texttt{ub}, \texttt{mean}, \texttt{sterr},
    \texttt{level}.\\
  \item
    Setters: \texttt{lb}, \texttt{ub}, \texttt{mean}, \texttt{sterr},
    \texttt{level}. Be sure to validate the resulting \texttt{waldCI}.\\
  \item
    A \texttt{contains} method, returning a logical of whether a value
    is within a CI.\\
  \item
    An \texttt{overlap} method, that takes in two \texttt{waldCI}'s, and
    returns a logical of whether the two confidence intervals overlap.\\
  \item
    \texttt{as.numeric} to return \texttt{c(lb,\ ub)}. (Hint: The second
    argument of \texttt{setGeneric} is not needed when an existing s3
    function uses the \texttt{.Primitive} function.)\\
  \item
    \texttt{transformCI} which takes in a \texttt{function} and a
    \texttt{waldCI}, and returns the transformed \texttt{waldCI} object.
    Warn the user that only monotonic functions make sense.\\
  \end{enumerate}
\item
  Use your \texttt{waldCI} class to create three objects:

  \begin{itemize}
  \tightlist
  \item
    \texttt{ci1}: (17.2, 24.7), 95\%\\
  \item
    \texttt{ci2}: mean: 13, standard error: 2.5, 99\%\\
  \item
    \texttt{ci3}: (27.43, 39.22), 75\%
  \end{itemize}

  Evaluate the following code:
\item
  Show that your validator does not allow the creation of invalid
  confidence intervals:
\end{enumerate}

\begin{itemize}
\tightlist
\item
  negative standard error\\
\item
  lb \textgreater{} ub\\
\item
  Infinite bounds\\
\item
  invalid use of the setters
\end{itemize}

(Infinite confidence bounds are of course not truely invalid, but we're
just going to ignore them for this case.)

Note that there are a lot of choices to be made here. What are you going
to store in the class? How are you going to store them (what object
types)? How are you going to enforce the function in \texttt{transform}
being monotonic?

There is no right answer to those questions. Make the best decision you
can, and don't be afraid to change it if your decision causes unforeseen
difficulties.

You may not use any existing R functions or packages that would
trivialize this assignment. (E.g. if you found an existing package that
does this, or found a function that checks for overlap between two CIs,
that is not able to be used.)

\subsection{Problem 3 - plotly}\label{problem-3---plotly}

Repeat {[}problem set 4, question 3{]} using plotly.

There is no expectation that you produce the exact same plots as last
time. You may of course use your plots as last time, or the ones from
the problem set 4 solutions, as inspiration for these plots.

These will be graded similar to last time:

\begin{enumerate}
\def\labelenumi{\arabic{enumi}.}
\item
  Is the type of graph \& choice of variables appropriate to answer the
  question?\\
\item
  Is the graph clear and easy to interpret?\\
\item ~
  \subsection{Is the graph publication
  ready?}\label{is-the-graph-publication-ready}
\end{enumerate}

\subsection{GitHub Link}\label{github-link}

\begin{itemize}
\tightlist
\item
  Repo: \url{https://github.com/brynnwoolley/STATS-506\#}
\end{itemize}

\begin{center}\rule{0.5\linewidth}{0.5pt}\end{center}




\end{document}
