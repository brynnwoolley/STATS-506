% Options for packages loaded elsewhere
% Options for packages loaded elsewhere
\PassOptionsToPackage{unicode}{hyperref}
\PassOptionsToPackage{hyphens}{url}
\PassOptionsToPackage{dvipsnames,svgnames,x11names}{xcolor}
%
\documentclass[
  letterpaper,
  DIV=11,
  numbers=noendperiod]{scrartcl}
\usepackage{xcolor}
\usepackage{amsmath,amssymb}
\setcounter{secnumdepth}{-\maxdimen} % remove section numbering
\usepackage{iftex}
\ifPDFTeX
  \usepackage[T1]{fontenc}
  \usepackage[utf8]{inputenc}
  \usepackage{textcomp} % provide euro and other symbols
\else % if luatex or xetex
  \usepackage{unicode-math} % this also loads fontspec
  \defaultfontfeatures{Scale=MatchLowercase}
  \defaultfontfeatures[\rmfamily]{Ligatures=TeX,Scale=1}
\fi
\usepackage{lmodern}
\ifPDFTeX\else
  % xetex/luatex font selection
\fi
% Use upquote if available, for straight quotes in verbatim environments
\IfFileExists{upquote.sty}{\usepackage{upquote}}{}
\IfFileExists{microtype.sty}{% use microtype if available
  \usepackage[]{microtype}
  \UseMicrotypeSet[protrusion]{basicmath} % disable protrusion for tt fonts
}{}
\makeatletter
\@ifundefined{KOMAClassName}{% if non-KOMA class
  \IfFileExists{parskip.sty}{%
    \usepackage{parskip}
  }{% else
    \setlength{\parindent}{0pt}
    \setlength{\parskip}{6pt plus 2pt minus 1pt}}
}{% if KOMA class
  \KOMAoptions{parskip=half}}
\makeatother
% Make \paragraph and \subparagraph free-standing
\makeatletter
\ifx\paragraph\undefined\else
  \let\oldparagraph\paragraph
  \renewcommand{\paragraph}{
    \@ifstar
      \xxxParagraphStar
      \xxxParagraphNoStar
  }
  \newcommand{\xxxParagraphStar}[1]{\oldparagraph*{#1}\mbox{}}
  \newcommand{\xxxParagraphNoStar}[1]{\oldparagraph{#1}\mbox{}}
\fi
\ifx\subparagraph\undefined\else
  \let\oldsubparagraph\subparagraph
  \renewcommand{\subparagraph}{
    \@ifstar
      \xxxSubParagraphStar
      \xxxSubParagraphNoStar
  }
  \newcommand{\xxxSubParagraphStar}[1]{\oldsubparagraph*{#1}\mbox{}}
  \newcommand{\xxxSubParagraphNoStar}[1]{\oldsubparagraph{#1}\mbox{}}
\fi
\makeatother

\usepackage{color}
\usepackage{fancyvrb}
\newcommand{\VerbBar}{|}
\newcommand{\VERB}{\Verb[commandchars=\\\{\}]}
\DefineVerbatimEnvironment{Highlighting}{Verbatim}{commandchars=\\\{\}}
% Add ',fontsize=\small' for more characters per line
\usepackage{framed}
\definecolor{shadecolor}{RGB}{241,243,245}
\newenvironment{Shaded}{\begin{snugshade}}{\end{snugshade}}
\newcommand{\AlertTok}[1]{\textcolor[rgb]{0.68,0.00,0.00}{#1}}
\newcommand{\AnnotationTok}[1]{\textcolor[rgb]{0.37,0.37,0.37}{#1}}
\newcommand{\AttributeTok}[1]{\textcolor[rgb]{0.40,0.45,0.13}{#1}}
\newcommand{\BaseNTok}[1]{\textcolor[rgb]{0.68,0.00,0.00}{#1}}
\newcommand{\BuiltInTok}[1]{\textcolor[rgb]{0.00,0.23,0.31}{#1}}
\newcommand{\CharTok}[1]{\textcolor[rgb]{0.13,0.47,0.30}{#1}}
\newcommand{\CommentTok}[1]{\textcolor[rgb]{0.37,0.37,0.37}{#1}}
\newcommand{\CommentVarTok}[1]{\textcolor[rgb]{0.37,0.37,0.37}{\textit{#1}}}
\newcommand{\ConstantTok}[1]{\textcolor[rgb]{0.56,0.35,0.01}{#1}}
\newcommand{\ControlFlowTok}[1]{\textcolor[rgb]{0.00,0.23,0.31}{\textbf{#1}}}
\newcommand{\DataTypeTok}[1]{\textcolor[rgb]{0.68,0.00,0.00}{#1}}
\newcommand{\DecValTok}[1]{\textcolor[rgb]{0.68,0.00,0.00}{#1}}
\newcommand{\DocumentationTok}[1]{\textcolor[rgb]{0.37,0.37,0.37}{\textit{#1}}}
\newcommand{\ErrorTok}[1]{\textcolor[rgb]{0.68,0.00,0.00}{#1}}
\newcommand{\ExtensionTok}[1]{\textcolor[rgb]{0.00,0.23,0.31}{#1}}
\newcommand{\FloatTok}[1]{\textcolor[rgb]{0.68,0.00,0.00}{#1}}
\newcommand{\FunctionTok}[1]{\textcolor[rgb]{0.28,0.35,0.67}{#1}}
\newcommand{\ImportTok}[1]{\textcolor[rgb]{0.00,0.46,0.62}{#1}}
\newcommand{\InformationTok}[1]{\textcolor[rgb]{0.37,0.37,0.37}{#1}}
\newcommand{\KeywordTok}[1]{\textcolor[rgb]{0.00,0.23,0.31}{\textbf{#1}}}
\newcommand{\NormalTok}[1]{\textcolor[rgb]{0.00,0.23,0.31}{#1}}
\newcommand{\OperatorTok}[1]{\textcolor[rgb]{0.37,0.37,0.37}{#1}}
\newcommand{\OtherTok}[1]{\textcolor[rgb]{0.00,0.23,0.31}{#1}}
\newcommand{\PreprocessorTok}[1]{\textcolor[rgb]{0.68,0.00,0.00}{#1}}
\newcommand{\RegionMarkerTok}[1]{\textcolor[rgb]{0.00,0.23,0.31}{#1}}
\newcommand{\SpecialCharTok}[1]{\textcolor[rgb]{0.37,0.37,0.37}{#1}}
\newcommand{\SpecialStringTok}[1]{\textcolor[rgb]{0.13,0.47,0.30}{#1}}
\newcommand{\StringTok}[1]{\textcolor[rgb]{0.13,0.47,0.30}{#1}}
\newcommand{\VariableTok}[1]{\textcolor[rgb]{0.07,0.07,0.07}{#1}}
\newcommand{\VerbatimStringTok}[1]{\textcolor[rgb]{0.13,0.47,0.30}{#1}}
\newcommand{\WarningTok}[1]{\textcolor[rgb]{0.37,0.37,0.37}{\textit{#1}}}

\usepackage{longtable,booktabs,array}
\usepackage{calc} % for calculating minipage widths
% Correct order of tables after \paragraph or \subparagraph
\usepackage{etoolbox}
\makeatletter
\patchcmd\longtable{\par}{\if@noskipsec\mbox{}\fi\par}{}{}
\makeatother
% Allow footnotes in longtable head/foot
\IfFileExists{footnotehyper.sty}{\usepackage{footnotehyper}}{\usepackage{footnote}}
\makesavenoteenv{longtable}
\usepackage{graphicx}
\makeatletter
\newsavebox\pandoc@box
\newcommand*\pandocbounded[1]{% scales image to fit in text height/width
  \sbox\pandoc@box{#1}%
  \Gscale@div\@tempa{\textheight}{\dimexpr\ht\pandoc@box+\dp\pandoc@box\relax}%
  \Gscale@div\@tempb{\linewidth}{\wd\pandoc@box}%
  \ifdim\@tempb\p@<\@tempa\p@\let\@tempa\@tempb\fi% select the smaller of both
  \ifdim\@tempa\p@<\p@\scalebox{\@tempa}{\usebox\pandoc@box}%
  \else\usebox{\pandoc@box}%
  \fi%
}
% Set default figure placement to htbp
\def\fps@figure{htbp}
\makeatother





\setlength{\emergencystretch}{3em} % prevent overfull lines

\providecommand{\tightlist}{%
  \setlength{\itemsep}{0pt}\setlength{\parskip}{0pt}}



 


\usepackage{amsmath}
\KOMAoption{captions}{tableheading}
\makeatletter
\@ifpackageloaded{caption}{}{\usepackage{caption}}
\AtBeginDocument{%
\ifdefined\contentsname
  \renewcommand*\contentsname{Table of contents}
\else
  \newcommand\contentsname{Table of contents}
\fi
\ifdefined\listfigurename
  \renewcommand*\listfigurename{List of Figures}
\else
  \newcommand\listfigurename{List of Figures}
\fi
\ifdefined\listtablename
  \renewcommand*\listtablename{List of Tables}
\else
  \newcommand\listtablename{List of Tables}
\fi
\ifdefined\figurename
  \renewcommand*\figurename{Figure}
\else
  \newcommand\figurename{Figure}
\fi
\ifdefined\tablename
  \renewcommand*\tablename{Table}
\else
  \newcommand\tablename{Table}
\fi
}
\@ifpackageloaded{float}{}{\usepackage{float}}
\floatstyle{ruled}
\@ifundefined{c@chapter}{\newfloat{codelisting}{h}{lop}}{\newfloat{codelisting}{h}{lop}[chapter]}
\floatname{codelisting}{Listing}
\newcommand*\listoflistings{\listof{codelisting}{List of Listings}}
\makeatother
\makeatletter
\makeatother
\makeatletter
\@ifpackageloaded{caption}{}{\usepackage{caption}}
\@ifpackageloaded{subcaption}{}{\usepackage{subcaption}}
\makeatother
\usepackage{bookmark}
\IfFileExists{xurl.sty}{\usepackage{xurl}}{} % add URL line breaks if available
\urlstyle{same}
\hypersetup{
  pdftitle={Problem Set 2},
  pdfauthor={Brynn Woolley},
  colorlinks=true,
  linkcolor={blue},
  filecolor={Maroon},
  citecolor={Blue},
  urlcolor={Blue},
  pdfcreator={LaTeX via pandoc}}


\title{Problem Set 2}
\author{Brynn Woolley}
\date{}
\begin{document}
\maketitle


\subsection{Problem 1 - Modified Random
walk}\label{problem-1---modified-random-walk}

Consider a 1-dimensional random walk with the following rules:

\begin{enumerate}
\def\labelenumi{\arabic{enumi}.}
\tightlist
\item
  Start at 0.\\
\item
  At each step, move +1 or -1 with 50/50 probability.\\
\item
  If +1 is chosen, 5\% of the time move +10 instead.\\
\item
  If -1 is chosen, 20\% of the time move -3 instead.\\
\item
  Repeat steps 2-4 (n) times.
\end{enumerate}

(Note that if the +10 is chosen, it's not +1 then +10, it is just +10.)

Write a function to determine the end position of this random walk.

The input and output should be:\\
* Input: The number of steps\\
* Output: The final position of the walk

\begin{verbatim}
> random_walk(10)  
 [1]  4  

> random_walk(10)  
 [1]  -11  
\end{verbatim}

We're going to implement this in different ways and compare them.

\begin{enumerate}
\def\labelenumi{\arabic{enumi}.}
\item
  Implement the random walk in these three versions:

  \begin{itemize}
  \item
    Version 1: using a loop.\\
  \item
    Version 2: using built-in R vectorized functions. (Using no loops.)
    (Hint: Does the order of steps matter?)\\
  \item
    Version 3: Implement the random walk using one of the
    ``\texttt{apply}'' functions.\\
  \end{itemize}
\end{enumerate}

\begin{Shaded}
\begin{Highlighting}[]
\CommentTok{\# Helper Function}
\NormalTok{random\_walk\_fixed }\OtherTok{\textless{}{-}} \ControlFlowTok{function}\NormalTok{(n, }\AttributeTok{draws =} \ConstantTok{NULL}\NormalTok{) \{}
  \ControlFlowTok{if}\NormalTok{ (}\FunctionTok{is.null}\NormalTok{(draws)) \{}
\NormalTok{    draws }\OtherTok{\textless{}{-}} \FunctionTok{runif}\NormalTok{(}\DecValTok{2} \SpecialCharTok{*}\NormalTok{ n)}
\NormalTok{  \}}
  
\NormalTok{  pos }\OtherTok{\textless{}{-}} \DecValTok{0}
  \ControlFlowTok{for}\NormalTok{ (i }\ControlFlowTok{in} \DecValTok{1}\SpecialCharTok{:}\NormalTok{n) \{}
\NormalTok{    dir\_draw }\OtherTok{\textless{}{-}}\NormalTok{ draws[(}\DecValTok{2} \SpecialCharTok{*}\NormalTok{ i }\SpecialCharTok{{-}} \DecValTok{1}\NormalTok{)]}
\NormalTok{    size\_draw }\OtherTok{\textless{}{-}}\NormalTok{ draws[(}\DecValTok{2} \SpecialCharTok{*}\NormalTok{ i)]}
    
    \ControlFlowTok{if}\NormalTok{ (dir\_draw }\SpecialCharTok{\textless{}} \FloatTok{0.5}\NormalTok{) \{}
\NormalTok{      pos }\OtherTok{\textless{}{-}}\NormalTok{ pos }\SpecialCharTok{+} \FunctionTok{ifelse}\NormalTok{(size\_draw }\SpecialCharTok{\textless{}} \FloatTok{0.95}\NormalTok{, }\DecValTok{1}\NormalTok{, }\DecValTok{10}\NormalTok{)}
\NormalTok{    \} }\ControlFlowTok{else}\NormalTok{ \{}
\NormalTok{      pos }\OtherTok{\textless{}{-}}\NormalTok{ pos }\SpecialCharTok{+} \FunctionTok{ifelse}\NormalTok{(size\_draw }\SpecialCharTok{\textless{}} \FloatTok{0.80}\NormalTok{, }\SpecialCharTok{{-}}\DecValTok{1}\NormalTok{, }\SpecialCharTok{{-}}\DecValTok{3}\NormalTok{)}
\NormalTok{    \}}
\NormalTok{  \}}
\NormalTok{  pos}
\NormalTok{\}}


\CommentTok{\# Method 1}
\NormalTok{random\_walk1 }\OtherTok{\textless{}{-}} \ControlFlowTok{function}\NormalTok{(n, draws) \{}
\NormalTok{  pos }\OtherTok{\textless{}{-}} \DecValTok{0}
  \ControlFlowTok{for}\NormalTok{ (i }\ControlFlowTok{in} \DecValTok{1}\SpecialCharTok{:}\NormalTok{n) \{}
\NormalTok{    dir\_draw }\OtherTok{\textless{}{-}}\NormalTok{ draws[(}\DecValTok{2} \SpecialCharTok{*}\NormalTok{ i }\SpecialCharTok{{-}} \DecValTok{1}\NormalTok{)]}
\NormalTok{    size\_draw }\OtherTok{\textless{}{-}}\NormalTok{ draws[(}\DecValTok{2} \SpecialCharTok{*}\NormalTok{ i)]}
    \ControlFlowTok{if}\NormalTok{ (dir\_draw }\SpecialCharTok{\textless{}} \FloatTok{0.5}\NormalTok{) \{}
\NormalTok{      pos }\OtherTok{\textless{}{-}}\NormalTok{ pos }\SpecialCharTok{+} \FunctionTok{ifelse}\NormalTok{(size\_draw }\SpecialCharTok{\textless{}} \FloatTok{0.95}\NormalTok{, }\DecValTok{1}\NormalTok{, }\DecValTok{10}\NormalTok{)}
\NormalTok{    \} }\ControlFlowTok{else}\NormalTok{ \{}
\NormalTok{      pos }\OtherTok{\textless{}{-}}\NormalTok{ pos }\SpecialCharTok{+} \FunctionTok{ifelse}\NormalTok{(size\_draw }\SpecialCharTok{\textless{}} \FloatTok{0.80}\NormalTok{, }\SpecialCharTok{{-}}\DecValTok{1}\NormalTok{, }\SpecialCharTok{{-}}\DecValTok{3}\NormalTok{)}
\NormalTok{    \}}
\NormalTok{  \}}
\NormalTok{  pos}
\NormalTok{\}}

\CommentTok{\# Method 2}
\NormalTok{random\_walk2 }\OtherTok{\textless{}{-}} \ControlFlowTok{function}\NormalTok{(n, draws) \{}
\NormalTok{  dirs }\OtherTok{\textless{}{-}}\NormalTok{ draws[}\FunctionTok{seq}\NormalTok{(}\DecValTok{1}\NormalTok{, }\DecValTok{2} \SpecialCharTok{*}\NormalTok{ n, }\AttributeTok{by =} \DecValTok{2}\NormalTok{)]}
\NormalTok{  sizes }\OtherTok{\textless{}{-}}\NormalTok{ draws[}\FunctionTok{seq}\NormalTok{(}\DecValTok{2}\NormalTok{, }\DecValTok{2} \SpecialCharTok{*}\NormalTok{ n, }\AttributeTok{by =} \DecValTok{2}\NormalTok{)]}
\NormalTok{  steps }\OtherTok{\textless{}{-}} \FunctionTok{ifelse}\NormalTok{(}
\NormalTok{    dirs }\SpecialCharTok{\textless{}} \FloatTok{0.5}\NormalTok{,}
    \FunctionTok{ifelse}\NormalTok{(sizes }\SpecialCharTok{\textless{}} \FloatTok{0.95}\NormalTok{, }\DecValTok{1}\NormalTok{, }\DecValTok{10}\NormalTok{),}
    \FunctionTok{ifelse}\NormalTok{(sizes }\SpecialCharTok{\textless{}} \FloatTok{0.80}\NormalTok{, }\SpecialCharTok{{-}}\DecValTok{1}\NormalTok{, }\SpecialCharTok{{-}}\DecValTok{3}\NormalTok{)}
\NormalTok{  )}
  \FunctionTok{sum}\NormalTok{(steps)}
\NormalTok{\}}

\CommentTok{\# Method 3}
\NormalTok{random\_walk3 }\OtherTok{\textless{}{-}} \ControlFlowTok{function}\NormalTok{(n, draws) \{}
  \FunctionTok{sum}\NormalTok{(}\FunctionTok{sapply}\NormalTok{(}\DecValTok{1}\SpecialCharTok{:}\NormalTok{n, }\ControlFlowTok{function}\NormalTok{(i) \{}
\NormalTok{    dir\_draw }\OtherTok{\textless{}{-}}\NormalTok{ draws[(}\DecValTok{2} \SpecialCharTok{*}\NormalTok{ i }\SpecialCharTok{{-}} \DecValTok{1}\NormalTok{)]}
\NormalTok{    size\_draw }\OtherTok{\textless{}{-}}\NormalTok{ draws[(}\DecValTok{2} \SpecialCharTok{*}\NormalTok{ i)]}
    \ControlFlowTok{if}\NormalTok{ (dir\_draw }\SpecialCharTok{\textless{}} \FloatTok{0.5}\NormalTok{) \{}
      \FunctionTok{ifelse}\NormalTok{(size\_draw }\SpecialCharTok{\textless{}} \FloatTok{0.95}\NormalTok{, }\DecValTok{1}\NormalTok{, }\DecValTok{10}\NormalTok{)}
\NormalTok{    \} }\ControlFlowTok{else}\NormalTok{ \{}
      \FunctionTok{ifelse}\NormalTok{(size\_draw }\SpecialCharTok{\textless{}} \FloatTok{0.80}\NormalTok{, }\SpecialCharTok{{-}}\DecValTok{1}\NormalTok{, }\SpecialCharTok{{-}}\DecValTok{3}\NormalTok{)}
\NormalTok{    \}}
\NormalTok{  \}))}
\NormalTok{\}}
\end{Highlighting}
\end{Shaded}

\begin{verbatim}
    Demonstrate that all versions work by running the following:

    random_walk1(10)\
    random_walk2(10)\
    random_walk3(10)\
    random_walk1(1000)\
    random_walk2(1000)\
    random_walk3(1000)
\end{verbatim}

\begin{enumerate}
\def\labelenumi{\arabic{enumi}.}
\setcounter{enumi}{1}
\tightlist
\item
  Demonstrate that the three versions can give the same result. Show
  this for both \texttt{n=10} and \texttt{n=1000}. (You will need to add
  a way to control the randomization.)\\
\end{enumerate}

\begin{Shaded}
\begin{Highlighting}[]
\FunctionTok{set.seed}\NormalTok{(}\DecValTok{123}\NormalTok{)}
\NormalTok{draws }\OtherTok{\textless{}{-}} \FunctionTok{runif}\NormalTok{(}\DecValTok{2} \SpecialCharTok{*} \DecValTok{10}\NormalTok{)}
\FunctionTok{random\_walk1}\NormalTok{(}\DecValTok{10}\NormalTok{, draws)}
\end{Highlighting}
\end{Shaded}

\begin{verbatim}
[1] 7
\end{verbatim}

\begin{Shaded}
\begin{Highlighting}[]
\FunctionTok{random\_walk2}\NormalTok{(}\DecValTok{10}\NormalTok{, draws)}
\end{Highlighting}
\end{Shaded}

\begin{verbatim}
[1] 7
\end{verbatim}

\begin{Shaded}
\begin{Highlighting}[]
\FunctionTok{random\_walk3}\NormalTok{(}\DecValTok{10}\NormalTok{, draws)}
\end{Highlighting}
\end{Shaded}

\begin{verbatim}
[1] 7
\end{verbatim}

\begin{Shaded}
\begin{Highlighting}[]
\FunctionTok{set.seed}\NormalTok{(}\DecValTok{123}\NormalTok{)}
\NormalTok{draws }\OtherTok{\textless{}{-}} \FunctionTok{runif}\NormalTok{(}\DecValTok{2} \SpecialCharTok{*} \DecValTok{1000}\NormalTok{)}
\FunctionTok{random\_walk1}\NormalTok{(}\DecValTok{1000}\NormalTok{, draws)}
\end{Highlighting}
\end{Shaded}

\begin{verbatim}
[1] 78
\end{verbatim}

\begin{Shaded}
\begin{Highlighting}[]
\FunctionTok{random\_walk2}\NormalTok{(}\DecValTok{1000}\NormalTok{, draws)}
\end{Highlighting}
\end{Shaded}

\begin{verbatim}
[1] 78
\end{verbatim}

\begin{Shaded}
\begin{Highlighting}[]
\FunctionTok{random\_walk3}\NormalTok{(}\DecValTok{1000}\NormalTok{, draws)}
\end{Highlighting}
\end{Shaded}

\begin{verbatim}
[1] 78
\end{verbatim}

\begin{enumerate}
\def\labelenumi{\arabic{enumi}.}
\setcounter{enumi}{2}
\tightlist
\item
  Use the microbenchmark package to clearly demonstrate the speed of the
  implementations. Compare performance with a low input (1,000) and a
  large input (100,000). Discuss the results.\\
\end{enumerate}

\begin{Shaded}
\begin{Highlighting}[]
\FunctionTok{library}\NormalTok{(microbenchmark)}

\CommentTok{\# low input: 1,000 steps}
\FunctionTok{set.seed}\NormalTok{(}\DecValTok{123}\NormalTok{)}
\NormalTok{draws\_1k }\OtherTok{\textless{}{-}} \FunctionTok{runif}\NormalTok{(}\DecValTok{2} \SpecialCharTok{*} \DecValTok{1000}\NormalTok{)}
\NormalTok{bench\_1k }\OtherTok{\textless{}{-}} \FunctionTok{microbenchmark}\NormalTok{(}
  \AttributeTok{loop =} \FunctionTok{random\_walk1}\NormalTok{(}\DecValTok{1000}\NormalTok{, draws\_1k),}
  \AttributeTok{vectorized =} \FunctionTok{random\_walk2}\NormalTok{(}\DecValTok{1000}\NormalTok{, draws\_1k),}
  \AttributeTok{apply =} \FunctionTok{random\_walk3}\NormalTok{(}\DecValTok{1000}\NormalTok{, draws\_1k),}
  \AttributeTok{times =} \DecValTok{100}
\NormalTok{)}
\FunctionTok{print}\NormalTok{(bench\_1k)}
\end{Highlighting}
\end{Shaded}

\begin{verbatim}
Unit: microseconds
       expr    min     lq     mean  median      uq    max neval
       loop 1668.7 1842.1 2066.049 1886.50 2007.45 6622.1   100
 vectorized  180.4  202.9  241.351  222.05  254.25  442.0   100
      apply 2824.0 2993.9 3420.793 3111.40 3269.95 9757.8   100
\end{verbatim}

\begin{Shaded}
\begin{Highlighting}[]
\CommentTok{\# large input: 100,000 steps}
\FunctionTok{set.seed}\NormalTok{(}\DecValTok{123}\NormalTok{)}
\NormalTok{draws\_100k }\OtherTok{\textless{}{-}} \FunctionTok{runif}\NormalTok{(}\DecValTok{2} \SpecialCharTok{*} \DecValTok{100000}\NormalTok{)}
\NormalTok{bench\_100k }\OtherTok{\textless{}{-}} \FunctionTok{microbenchmark}\NormalTok{(}
  \AttributeTok{loop =} \FunctionTok{random\_walk1}\NormalTok{(}\DecValTok{100000}\NormalTok{, draws\_100k),}
  \AttributeTok{vectorized =} \FunctionTok{random\_walk2}\NormalTok{(}\DecValTok{100000}\NormalTok{, draws\_100k),}
  \AttributeTok{apply =} \FunctionTok{random\_walk3}\NormalTok{(}\DecValTok{100000}\NormalTok{, draws\_100k),}
  \AttributeTok{times =} \DecValTok{10}
\NormalTok{)}
\FunctionTok{print}\NormalTok{(bench\_100k)}
\end{Highlighting}
\end{Shaded}

\begin{verbatim}
Unit: milliseconds
       expr      min       lq      mean   median       uq      max neval
       loop 191.7225 249.1976 277.00601 273.2877 289.2033 356.0243    10
 vectorized  13.3627  17.0843  17.55125  17.6831  19.0742  22.1090    10
      apply 392.1293 445.2162 491.57884 464.8003 486.8995 718.5217    10
\end{verbatim}

\textbf{Discussion:}

The benchmarking results show that the \textbf{vectorized implementation
is the fastest}. For 1,000 steps, the vectorized method ran about 10
times faster than the loop and more than 10 times faster than the apply
method. For 100,000 steps, the vectorized method was about 15 times
faster than the loop and roughly 25 times faster than apply. The loop
implementation performed reasonably well but was consistently slower
than vectorized. The apply implementation was the slowest due to the
overhead of repeated function calls.

In summary:

- Vectorized (fastest, most efficient for large n)

- Loop (moderate performance, simpler to read)

- Apply (slowest, avoid for performance)

\begin{enumerate}
\def\labelenumi{\arabic{enumi}.}
\setcounter{enumi}{3}
\tightlist
\item
  What is the probability that the random walk ends at 0 if the number
  of steps is 10? 100? 1000? Defend your answers with evidence based
  upon a Monte Carlo simulation.
\end{enumerate}

\begin{Shaded}
\begin{Highlighting}[]
\FunctionTok{set.seed}\NormalTok{(}\DecValTok{123}\NormalTok{)}

\CommentTok{\# Monte Carlo simulation}
\NormalTok{simulate\_prob\_zero }\OtherTok{\textless{}{-}} \ControlFlowTok{function}\NormalTok{(n, }\AttributeTok{sims =} \DecValTok{100000}\NormalTok{) \{}
\NormalTok{  zeros }\OtherTok{\textless{}{-}} \FunctionTok{replicate}\NormalTok{(sims, \{}
\NormalTok{    draws }\OtherTok{\textless{}{-}} \FunctionTok{runif}\NormalTok{(}\DecValTok{2} \SpecialCharTok{*}\NormalTok{ n)}
    \FunctionTok{random\_walk2}\NormalTok{(n, draws)}
\NormalTok{  \})}
  \FunctionTok{mean}\NormalTok{(zeros }\SpecialCharTok{==} \DecValTok{0}\NormalTok{)}
\NormalTok{\}}

\CommentTok{\# Run for different step sizes}
\NormalTok{prob\_10   }\OtherTok{\textless{}{-}} \FunctionTok{simulate\_prob\_zero}\NormalTok{(}\DecValTok{10}\NormalTok{)}
\NormalTok{prob\_100  }\OtherTok{\textless{}{-}} \FunctionTok{simulate\_prob\_zero}\NormalTok{(}\DecValTok{100}\NormalTok{)}
\NormalTok{prob\_1000 }\OtherTok{\textless{}{-}} \FunctionTok{simulate\_prob\_zero}\NormalTok{(}\DecValTok{1000}\NormalTok{)}

\NormalTok{prob\_10}
\end{Highlighting}
\end{Shaded}

\begin{verbatim}
[1] 0.1323
\end{verbatim}

\begin{Shaded}
\begin{Highlighting}[]
\NormalTok{prob\_100}
\end{Highlighting}
\end{Shaded}

\begin{verbatim}
[1] 0.01921
\end{verbatim}

\begin{Shaded}
\begin{Highlighting}[]
\NormalTok{prob\_1000}
\end{Highlighting}
\end{Shaded}

\begin{verbatim}
[1] 0.00575
\end{verbatim}

\subsection{Problem 2 - Mean of Mixture of
Distributions}\label{problem-2---mean-of-mixture-of-distributions}

The number of cars passing an intersection is a classic example of a
Poisson distribution. At a particular intersection, Poisson is an
appropriate distribution most of the time, but during rush hours (hours
of 8am and 5pm) the distribution is really normally distributed with a
much higher mean.

Using a Monte Carlo simulation, estimate the average number of cars that
pass an intersection per day under the following assumptions:

\begin{itemize}
\tightlist
\item
  From midnight until 7 AM, the distribution of cars per hour is Poisson
  with mean 1.\\
\item
  From 9am to 4pm, the distribution of cars per hour is Poisson with
  mean 8.\\
\item
  From 6pm to 11pm, the distribution of cars per hour is Poisson with
  mean 12.\\
\item
  During rush hours (8am and 5pm), the distribution of cars per hour is
  Normal with mean 60 and variance 12
\end{itemize}

Accomplish this without using any loops.

(Hint: This can be done with extremely minimal code.)

\subsection{Problem 3 - Linear
Regression}\label{problem-3---linear-regression}

Use the following code to download the YouTube Superbowl commercials
data:

\begin{verbatim}
youtube <- read.csv('https://raw.githubusercontent.com/rfordatascience/tidytuesday/master/data/2021/2021-03-02/youtube.csv')  
\end{verbatim}

Information about this data can be found at
https://github.com/rfordatascience/tidytuesday/tree/main/data/2021/2021-03-02.
The research question for this project is to decide which of several
attributes, if any, is associated with increased YouTube engagement
metrics.

\begin{enumerate}
\def\labelenumi{\arabic{enumi}.}
\tightlist
\item
  Often in data analysis, we need to de-identify it. This is more
  important for studies of people, but let's carry it out here. Remove
  any column that might uniquely identify a commercial. This includes
  but isn't limited to things like brand, any URLs, the YouTube channel,
  or when it was published.
\end{enumerate}

Report the dimensions of the data after removing these columns.\\
2. For each of the following variables, examine their distribution.
Determine whether i) The variable could be used as is as the outcome in
a linear regression model, ii) The variable can use a transformation
prior to being used as the outcome in a linear regression model, or iii)
The variable would not be appropriate to use as the outcome in a linear
regression model.\\
For each variable, report which category it falls in. If it requires a
transformation, carry such a transformation out and use that
transformation going forward.

\begin{verbatim}
* View counts  
* Like counts  
* Dislike counts  
* Favorite counts  
* Comment counts
\end{verbatim}

(Hint: At least the majority of these variables are appropriate to
use.)\\
3. For each variable in part b. that are appropriate, fit a linear
regression model predicting them based upon each of the seven binary
flags for characteristics of the ads, such as whether it is funny.
Control for year as a continuous covariate.

Discuss the results. Identify the direction of any statistically
significant results.\\
4. Consider only the outcome of view counts. Calculate (\hat{\beta})
manually (without using \texttt{lm}) by first creating a proper design
matrix, then using matrix algebra to estimate (\beta). Confirm that you
get the same result as \texttt{lm} did in part
c.~------------------------------------------------------------------------

\subsection{GitHub Link}\label{github-link}

\begin{itemize}
\tightlist
\item
  Repo: \url{https://github.com/brynnwoolley/STATS-506\#}
\end{itemize}

\begin{center}\rule{0.5\linewidth}{0.5pt}\end{center}




\end{document}
